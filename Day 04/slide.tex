

\documentclass{beamer}

\mode<presentation> {

% The Beamer class comes with a number of default slide themes
% which change the colors and layouts of slides. Below this is a list
% of all the themes, uncomment each in turn to see what they look like.

%\usetheme{default}
%\usetheme{AnnArbor}
%\usetheme{Antibes}
%\usetheme{Bergen}
%\usetheme{Berkeley}
%\usetheme{Berlin}
%\usetheme{Boadilla}
%\usetheme{CambridgeUS}
%\usetheme{Copenhagen}
%\usetheme{Darmstadt}
%\usetheme{Dresden}
%\usetheme{Frankfurt}
%\usetheme{Goettingen}
%\usetheme{Hannover}
%\usetheme{Ilmenau}
%\usetheme{JuanLesPins}
%\usetheme{Luebeck}
%\usetheme{Madrid}
%\usetheme{Malmoe}
%\usetheme{Marburg}
%\usetheme{Montpellier}
%\usetheme{PaloAlto}
%\usetheme{Pittsburgh}
%\usetheme{Rochester}
%\usetheme{Singapore}
%\usetheme{Szeged}
%\usetheme{Warsaw}

% As well as themes, the Beamer class has a number of color themes
% for any slide theme. Uncomment each of these in turn to see how it
% changes the colors of your current slide theme.

%\usecolortheme{albatross}
%\usecolortheme{beaver}
%\usecolortheme{beetle}
%\usecolortheme{crane}
%\usecolortheme{dolphin}
%\usecolortheme{dove}
%\usecolortheme{fly}
%\usecolortheme{lily}
%\usecolortheme{orchid}
%\usecolortheme{rose}
%\usecolortheme{seagull}
%\usecolortheme{seahorse}
%\usecolortheme{whale}
\usecolortheme{wolverine}

%\setbeamertemplate{footline} % To remove the footer line in all slides uncomment this line
%\setbeamertemplate{footline}[page number] % To replace the footer line in all slides with a simple slide count uncomment this line

%\setbeamertemplate{navigation symbols}{} % To remove the navigation symbols from the bottom of all slides uncomment this line
}

\usepackage{graphicx} % Allows including images
\usepackage{booktabs} % Allows the use of \toprule, \midrule and \bottomrule in tables


\usepackage{listings}
\lstset{language=Java,
                basicstyle=\footnotesize\ttfamily,
                keywordstyle=\footnotesize\color{blue}\ttfamily,
}

%----------------------------------------------------------------------------------------
%	TITLE PAGE
%----------------------------------------------------------------------------------------

\title[Arrays \& Iterations]{4.Arrays and iterations} % The short title appears at the bottom of every slide, the full title is only on the title page

\author{Sakib Abrar} % Your name
\institute[BUET] % Your institution as it will appear on the bottom of every slide, may be shorthand to save space
{
CSE\\~\\Bangladesh University of Engineering \& Technology \\ % Your institution for the title page
\medskip
\textit{sakib.cghs@gmail.com} % Your email address
}
\date{\today} % Date, can be changed to a custom date

\begin{document}

\begin{frame}
\titlepage % Print the title page as the first slide
\end{frame}

\begin{frame}
\frametitle{Overview} % Table of contents slide, comment this block out to remove it
\tableofcontents % Throughout your presentation, if you choose to use \section{} and \subsection{} commands, these will automatically be printed on this slide as an overview of your presentation
\end{frame}

%----------------------------------------------------------------------------------------
%	PRESENTATION SLIDES
%----------------------------------------------------------------------------------------

%------------------------------------------------
\section{Array basics}
%------------------------------------------------

\begin{frame}
\frametitle{What is an array?}
\begin{itemize}
\item A group of variables containing values that all have the same type.
\item Arrays are fixed‐length entities
\item In Java, arrays are objects, so they are considered reference types
\item But the elements of an array can be either primitive types or reference types
\end{itemize}
\end{frame}


%------------------------------------------------

\begin{frame}
\frametitle{More on array}
\begin{itemize}
\item We access the element of an array using the following syntax.\\
–name[index]\\
–“index” must be a nonnegative integer\\
–“index” can be int/byte/short/char but not long\\
\item In Java, every array knows its own length\\
\item The length information is maintained in a public final int member variable called length\\
\end{itemize}
\end{frame}

%------------------------------------------------
\section{Declaring and Creating Arrays}
%------------------------------------------------

\begin{frame}
\frametitle{Declaring and Creating Arrays}
\begin{itemize}
\item int c[ ] = new int [12]\\
–Here, “c” is a reference to an integer array\\
–“c” is now pointing to an array object holding 12 integers\\
–Like other objects arrays are created using “new” and are created in the heap\\
–“int c[ ]” represents both the data type and the variable name. Placing number here is a syntax error.\\
–int c[12]; // compiler error\\
\end{itemize}
\end{frame}

%------------------------------------------------
\section{Conditional Example}
%------------------------------------------------


\begin{frame}[fragile]{Example}
\textbf{Here is an example:}\\
\begin{columns}[T]
% code
\begin{column}{\textwidth}
\begin{lstlisting}
public class ArrayExamples {
    public static void main(String[] args) {
        int arr[] = new int[10];

        for (int idx = 0; idx < arr.length; idx++ ) {
            arr[idx] = idx * idx; // square
        }

        for (int idx = 0; idx < arr.length; idx++ ) {
            System.out.println(arr[idx]);
        }

    }
}
\end{lstlisting}
\end{column}
% description
\end{columns}
\end{frame}



%-----------------------------------------------
\section{Using an Array Initializer}
%-----------------------------------------------

\begin{frame}{Using an Array Initializer}
\begin{itemize}
\item We can also use an array initializer to create an array
–int arr[ ] = {10, 20, 30, 40, 50}\\
\item The length of the above array is 5
\item n[0] is initialized to 10, n[1] is initialized to 20, and so on\\
\item The compiler automatically performs a “new” operation taking the count information from the list and initializes the elements properly\\
\end{itemize}
\end{frame}


%-----------------------------------------------
\section{Multidimensional Arrays}
%------------------------------------------------

\begin{frame}[fragile]
\frametitle{Multidimensional Arrays}
\begin{itemize}
\item Can be termed as array of arrays.\\
\item int b[ ][ ] = new int[3][4];\\
–Length of first dimension = 3\\
•b.length equals 3\\
–Length of second dimension = 4\\
•b[0].length equals 4\\
\item int[ ][ ] b = new int[3][4];\\
–Here, the data type is more evident i.e. “int[ ][ ]”\\
\end{itemize}
\end{frame}

%------------------------------------------------

%-----------------------------------------------
\section{For-each loop}
%------------------------------------------------

\begin{frame}[fragile]
\frametitle{For-each loop}
\textbf{For each loops are more comfy with arrays:}\\
\begin{columns}[T]
% code
\begin{column}{\textwidth}
\begin{lstlisting}
public class ArrayExamples {
    public static void main(String[] args) {

        int arr[] = new int[10];
        for (int idx = 0; idx < arr.length; idx++ ) {
            arr[idx] = idx * idx;
        }
        for (int ele : arr) {
            System.out.println(ele);
        }

    }
}
\end{lstlisting}
\end{column}
% description
\end{columns}
\end{frame}

%------------------------------------------------


%-----------------------------------------------
\section{Array Exercise}
%------------------------------------------------

\begin{frame}
\frametitle{Array Exercise}
\textbf{For 20 students store their marks of 4 subjects using random number generator. Then you'll be asked to show gpa of any students through input. Calculate and print the gpa of that student.}\\
\end{frame}

%------------------------------------------------


%--------------------------------------------------

\begin{frame}
\Huge{\centerline{Take a break }}
\Huge{\centerline{You've learned enough already }}
\vspace{0.2in}
\Huge{\centerline{THE END }}
\end{frame}

%----------------------------------------------------------------------------------------

\end{document} 