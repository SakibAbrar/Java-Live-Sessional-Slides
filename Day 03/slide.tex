

\documentclass{beamer}

\mode<presentation> {

% The Beamer class comes with a number of default slide themes
% which change the colors and layouts of slides. Below this is a list
% of all the themes, uncomment each in turn to see what they look like.

%\usetheme{default}
%\usetheme{AnnArbor}
%\usetheme{Antibes}
%\usetheme{Bergen}
%\usetheme{Berkeley}
%\usetheme{Berlin}
%\usetheme{Boadilla}
%\usetheme{CambridgeUS}
%\usetheme{Copenhagen}
%\usetheme{Darmstadt}
%\usetheme{Dresden}
%\usetheme{Frankfurt}
%\usetheme{Goettingen}
%\usetheme{Hannover}
%\usetheme{Ilmenau}
%\usetheme{JuanLesPins}
%\usetheme{Luebeck}
%\usetheme{Madrid}
%\usetheme{Malmoe}
%\usetheme{Marburg}
%\usetheme{Montpellier}
%\usetheme{PaloAlto}
%\usetheme{Pittsburgh}
%\usetheme{Rochester}
%\usetheme{Singapore}
%\usetheme{Szeged}
%\usetheme{Warsaw}

% As well as themes, the Beamer class has a number of color themes
% for any slide theme. Uncomment each of these in turn to see how it
% changes the colors of your current slide theme.

%\usecolortheme{albatross}
%\usecolortheme{beaver}
%\usecolortheme{beetle}
%\usecolortheme{crane}
%\usecolortheme{dolphin}
%\usecolortheme{dove}
%\usecolortheme{fly}
%\usecolortheme{lily}
%\usecolortheme{orchid}
%\usecolortheme{rose}
%\usecolortheme{seagull}
%\usecolortheme{seahorse}
%\usecolortheme{whale}
\usecolortheme{wolverine}

%\setbeamertemplate{footline} % To remove the footer line in all slides uncomment this line
%\setbeamertemplate{footline}[page number] % To replace the footer line in all slides with a simple slide count uncomment this line

%\setbeamertemplate{navigation symbols}{} % To remove the navigation symbols from the bottom of all slides uncomment this line
}

\usepackage{graphicx} % Allows including images
\usepackage{booktabs} % Allows the use of \toprule, \midrule and \bottomrule in tables


\usepackage{listings}
\lstset{language=Java,
                basicstyle=\footnotesize\ttfamily,
                keywordstyle=\footnotesize\color{blue}\ttfamily,
}

%----------------------------------------------------------------------------------------
%	TITLE PAGE
%----------------------------------------------------------------------------------------

\title[Conditionals \& Loops]{3.Conditional statements and loops} % The short title appears at the bottom of every slide, the full title is only on the title page

\author{Sakib Abrar} % Your name
\institute[BUET] % Your institution as it will appear on the bottom of every slide, may be shorthand to save space
{
CSE\\~\\Bangladesh University of Engineering \& Technology \\ % Your institution for the title page
\medskip
\textit{sakib.cghs@gmail.com} % Your email address
}
\date{\today} % Date, can be changed to a custom date

\begin{document}

\begin{frame}
\titlepage % Print the title page as the first slide
\end{frame}

\begin{frame}
\frametitle{Overview} % Table of contents slide, comment this block out to remove it
\tableofcontents % Throughout your presentation, if you choose to use \section{} and \subsection{} commands, these will automatically be printed on this slide as an overview of your presentation
\end{frame}

%----------------------------------------------------------------------------------------
%	PRESENTATION SLIDES
%----------------------------------------------------------------------------------------

%------------------------------------------------
\section{Programming is all about logic}
%------------------------------------------------

\begin{frame}
\frametitle{Programming is all about logic}
A game or any other application mainly consist of lots of logic.\\~\\
You'll have to deal with the logic of your application.\\~\\
Figuring out which logic should go with your application is the main task.\\~\\
To implement logic we need some sort of basic tools for it. \\~\\
Conditional and loops are the core tool of logic for this.\\~\\


\end{frame}

%------------------------------------------------
\section{Conditional}
%------------------------------------------------

\begin{frame}
\frametitle{Conditional}
\centerline{\huge{ if $( some conditions) \{  some tasks \}$  }}
\vspace{0.4in}
It is what it sounds. You impose some sort of condition then if that happens you do something.\\~\\
Otherwise you do something else.
\end{frame}

%------------------------------------------------
\section{Conditional Example}
%------------------------------------------------


\begin{frame}[fragile]{Example}
\textbf{Here is an example of if:}\\
\begin{columns}[T]
% code
\begin{column}{\textwidth}
\begin{lstlisting}
public class ConditionalExamples {
    public static void main(String[] args) {
        int age;
        age = 10;

        if ( age < 18 ) {
            System.out.println("You are not allowed to view this content");
        }
    }
}
\end{lstlisting}
\end{column}
% description
\end{columns}
\vspace{0.4in} Then hit run and see!
\end{frame}



%-----------------------------------------------
\section{Data type declaration and assignment}
%-----------------------------------------------

\begin{frame}[fragile]{ Declaration and assignment}
\textbf{Here is an example of if and else:}\\
\begin{columns}[T]
% code
\begin{column}{\textwidth}
\begin{lstlisting}
public class ConditionalExamples {
    public static void main(String[] args) {
        int age;
        age = 10;

        if ( age < 18 ) {
            System.out.println("You are not allowed to view this content");
        }
        else {
            System.out.println("Welcome to our Wargame");
        }
    }
}
\end{lstlisting}
\end{column}
% description
\end{columns}
\vspace{0.4in} Then hit run and see!
\end{frame}


%-----------------------------------------------
\section{Multiple conditions}
%------------------------------------------------

\begin{frame}[fragile]
\frametitle{What about multiple condtions?}
\centerline{\huge{Use \&\& , $||$  ect  to combine conditions }}
\vspace{0.2in}
\begin{columns}[T]
% code
\begin{column}{\textwidth}
\begin{lstlisting}
public class ConditionalExamples {
    public static void main(String[] args) {
        int age;
        age = 10;

        if ( age > 18 && age < 60 ) {
            System.out.println("You are a matured person now");
        }
        else {
            System.out.println("You're either a baby or an old person");
        }
    }
}
\end{lstlisting}
\end{column}
% description
\end{columns}
\end{frame}


%-----------------------------------------------
\section{Multilevel conditional}
%------------------------------------------------

\begin{frame}[fragile]
\frametitle{Else if}
\centerline{\huge{Use else if for more conditions }}
\begin{columns}[T]
% code
\begin{column}{\textwidth}
\begin{lstlisting}
public class ConditionalExamples {
    public static void main(String[] args) {
        int age = 17;
        if ( age < 25 ) {
            System.out.println("You are young");
        }
        else if (age < 35) {
            System.out.println("You're having midlife crisis");
        }
        else if (age < 55) {
            System.out.println("You're slowly getting old");
        }
        else {
            System.out.println("You are old");
        }
    }
}
\end{lstlisting}
\end{column}
% description
\end{columns}
\end{frame}


%-----------------------------------------------
\section{Why we need loops}
%------------------------------------------------

\begin{frame}
\frametitle{Loops}
\centerline{\huge{Doing a task repeatedly is pretty boring.}}
\vspace{0.4in}
\centerline{\huge{Just use a loop to make your life easier.}}

\end{frame}


%-----------------------------------------------
\section{For Loop}
%------------------------------------------------

\begin{frame}[fragile]
\frametitle{For Loop}
\centerline{\Large{ $for(initialization; condition; increment) \{  some tasks \}$  }}
\vspace{0.2in}
\begin{itemize}
\item You take a variable and then for that variable you repeatedly do some tasks. Usually when the variable crosses some particular value you stop.
\item So first you initialize that variable then until a certain condition holds you do those tasks repeatedly.
\item Till that condition breaks you increment the value or you'll get stuck for ever.
\end{itemize}
\end{frame}

%-----------------------------------------------
\section{Loop Example}
%------------------------------------------------

\begin{frame}[fragile]
\frametitle{Examples}
\textbf{This will get printed 100 times:}\\
\begin{columns}[T]
% code
\begin{column}{\textwidth}
\begin{lstlisting}
public class LoopExamples {
    public static void main(String[] args) {

        for(int age = 1; age <= 100; age ++ ) {
            System.out.println("Happy " + age + "th Birthday");
        }

    }
}
\end{lstlisting}
\end{column}
% description
\end{columns}
\end{frame}


%--------------------------------------------------

%-----------------------------------------------
\section{You can use conditional within loops}
%------------------------------------------------

\begin{frame}[fragile]
\frametitle{Conditionals within loops}
\textbf{You can aslo use conditionals to manipulate loops}\\
\begin{columns}[T]
% code
\begin{column}{\textwidth}
\begin{lstlisting}
public class LoopExamples {
    public static void main(String[] args) {

        for(int age = 1; age <= 100; age ++ ) {
            if ( age % 2 == 0 ) {
                System.out.println("Happy Good Birthday");
            }
            else {
                System.out.println("Happy Pocha Birthday");
            }
        }
    }
}
\end{lstlisting}
\end{column}
% description
\end{columns}
\end{frame}


%--------------------------------------------------


%-----------------------------------------------
\section{break and continue statement}
%------------------------------------------------

\begin{frame}[fragile]
\frametitle{break and continue statement}
\textbf{break and continue statement can a good way to manipulate loops}\\
\begin{columns}[T]
% code
\begin{column}{\textwidth}
\begin{lstlisting}
public class LoopExamples {
    public static void main(String[] args) {

        for(int age = 1; age <= 100; age ++ ) {
            if ( age % 2 == 0 ) {
                continue;
            }
            System.out.println("Happy Birthday");
        }
    }
}
\end{lstlisting}
\end{column}
% description
\end{columns}
\end{frame}


%-----------------------------------------------
\section{Nested Loops and Conditionals}
%------------------------------------------------

\begin{frame}[fragile]
\frametitle{Nested Loops and Conditionals}
\textbf{You can alwasy use loops inside loops and conditionals inside conditionals}\\
\begin{columns}[T]
% code
\begin{column}{\textwidth}
\begin{lstlisting}
public class LoopExamples {
    public static void main(String[] args) {

        for(int age = 1; age <= 100; age ++ ) {
            for(int month = 1; month <= 12; month ++ ) {
                System.out.println("A:"+age+",M:"+month );
            }
        }

    }
}
\end{lstlisting}
\end{column}
% description
\end{columns}
\end{frame}

%--------------------------------------------------


\begin{frame}
\Huge{\centerline{The End}}
\end{frame}

%----------------------------------------------------------------------------------------

\end{document} 